%% Generated by Sphinx.
\def\sphinxdocclass{report}
\documentclass[letterpaper,10pt,english]{sphinxmanual}
\ifdefined\pdfpxdimen
   \let\sphinxpxdimen\pdfpxdimen\else\newdimen\sphinxpxdimen
\fi \sphinxpxdimen=.75bp\relax

\PassOptionsToPackage{warn}{textcomp}
\usepackage[utf8]{inputenc}
\ifdefined\DeclareUnicodeCharacter
% support both utf8 and utf8x syntaxes
  \ifdefined\DeclareUnicodeCharacterAsOptional
    \def\sphinxDUC#1{\DeclareUnicodeCharacter{"#1}}
  \else
    \let\sphinxDUC\DeclareUnicodeCharacter
  \fi
  \sphinxDUC{00A0}{\nobreakspace}
  \sphinxDUC{2500}{\sphinxunichar{2500}}
  \sphinxDUC{2502}{\sphinxunichar{2502}}
  \sphinxDUC{2514}{\sphinxunichar{2514}}
  \sphinxDUC{251C}{\sphinxunichar{251C}}
  \sphinxDUC{2572}{\textbackslash}
\fi
\usepackage{cmap}
\usepackage[T1]{fontenc}
\usepackage{amsmath,amssymb,amstext}
\usepackage{babel}



\usepackage{times}
\expandafter\ifx\csname T@LGR\endcsname\relax
\else
% LGR was declared as font encoding
  \substitutefont{LGR}{\rmdefault}{cmr}
  \substitutefont{LGR}{\sfdefault}{cmss}
  \substitutefont{LGR}{\ttdefault}{cmtt}
\fi
\expandafter\ifx\csname T@X2\endcsname\relax
  \expandafter\ifx\csname T@T2A\endcsname\relax
  \else
  % T2A was declared as font encoding
    \substitutefont{T2A}{\rmdefault}{cmr}
    \substitutefont{T2A}{\sfdefault}{cmss}
    \substitutefont{T2A}{\ttdefault}{cmtt}
  \fi
\else
% X2 was declared as font encoding
  \substitutefont{X2}{\rmdefault}{cmr}
  \substitutefont{X2}{\sfdefault}{cmss}
  \substitutefont{X2}{\ttdefault}{cmtt}
\fi


\usepackage[Bjarne]{fncychap}
\usepackage{sphinx}

\fvset{fontsize=\small}
\usepackage{geometry}


% Include hyperref last.
\usepackage{hyperref}
% Fix anchor placement for figures with captions.
\usepackage{hypcap}% it must be loaded after hyperref.
% Set up styles of URL: it should be placed after hyperref.
\urlstyle{same}

\usepackage{sphinxmessages}
\setcounter{tocdepth}{1}


% Jupyter Notebook code cell colors
\definecolor{nbsphinxin}{HTML}{307FC1}
\definecolor{nbsphinxout}{HTML}{BF5B3D}
\definecolor{nbsphinx-code-bg}{HTML}{F5F5F5}
\definecolor{nbsphinx-code-border}{HTML}{E0E0E0}
\definecolor{nbsphinx-stderr}{HTML}{FFDDDD}
% ANSI colors for output streams and traceback highlighting
\definecolor{ansi-black}{HTML}{3E424D}
\definecolor{ansi-black-intense}{HTML}{282C36}
\definecolor{ansi-red}{HTML}{E75C58}
\definecolor{ansi-red-intense}{HTML}{B22B31}
\definecolor{ansi-green}{HTML}{00A250}
\definecolor{ansi-green-intense}{HTML}{007427}
\definecolor{ansi-yellow}{HTML}{DDB62B}
\definecolor{ansi-yellow-intense}{HTML}{B27D12}
\definecolor{ansi-blue}{HTML}{208FFB}
\definecolor{ansi-blue-intense}{HTML}{0065CA}
\definecolor{ansi-magenta}{HTML}{D160C4}
\definecolor{ansi-magenta-intense}{HTML}{A03196}
\definecolor{ansi-cyan}{HTML}{60C6C8}
\definecolor{ansi-cyan-intense}{HTML}{258F8F}
\definecolor{ansi-white}{HTML}{C5C1B4}
\definecolor{ansi-white-intense}{HTML}{A1A6B2}
\definecolor{ansi-default-inverse-fg}{HTML}{FFFFFF}
\definecolor{ansi-default-inverse-bg}{HTML}{000000}

% Define an environment for non-plain-text code cell outputs (e.g. images)
\makeatletter
\newenvironment{nbsphinxfancyoutput}{%
    % Avoid fatal error with framed.sty if graphics too long to fit on one page
    \let\sphinxincludegraphics\nbsphinxincludegraphics
    \nbsphinx@image@maxheight\textheight
    \advance\nbsphinx@image@maxheight -2\fboxsep   % default \fboxsep 3pt
    \advance\nbsphinx@image@maxheight -2\fboxrule  % default \fboxrule 0.4pt
    \advance\nbsphinx@image@maxheight -\baselineskip
\def\nbsphinxfcolorbox{\spx@fcolorbox{nbsphinx-code-border}{white}}%
\def\FrameCommand{\nbsphinxfcolorbox\nbsphinxfancyaddprompt\@empty}%
\def\FirstFrameCommand{\nbsphinxfcolorbox\nbsphinxfancyaddprompt\sphinxVerbatim@Continues}%
\def\MidFrameCommand{\nbsphinxfcolorbox\sphinxVerbatim@Continued\sphinxVerbatim@Continues}%
\def\LastFrameCommand{\nbsphinxfcolorbox\sphinxVerbatim@Continued\@empty}%
\MakeFramed{\advance\hsize-\width\@totalleftmargin\z@\linewidth\hsize\@setminipage}%
\lineskip=1ex\lineskiplimit=1ex\raggedright%
}{\par\unskip\@minipagefalse\endMakeFramed}
\makeatother
\newbox\nbsphinxpromptbox
\def\nbsphinxfancyaddprompt{\ifvoid\nbsphinxpromptbox\else
    \kern\fboxrule\kern\fboxsep
    \copy\nbsphinxpromptbox
    \kern-\ht\nbsphinxpromptbox\kern-\dp\nbsphinxpromptbox
    \kern-\fboxsep\kern-\fboxrule\nointerlineskip
    \fi}
\newlength\nbsphinxcodecellspacing
\setlength{\nbsphinxcodecellspacing}{0pt}

% Define support macros for attaching opening and closing lines to notebooks
\newsavebox\nbsphinxbox
\makeatletter
\newcommand{\nbsphinxstartnotebook}[1]{%
    \par
    % measure needed space
    \setbox\nbsphinxbox\vtop{{#1\par}}
    % reserve some space at bottom of page, else start new page
    \needspace{\dimexpr2.5\baselineskip+\ht\nbsphinxbox+\dp\nbsphinxbox}
    % mimick vertical spacing from \section command
      \addpenalty\@secpenalty
      \@tempskipa 3.5ex \@plus 1ex \@minus .2ex\relax
      \addvspace\@tempskipa
      {\Large\@tempskipa\baselineskip
             \advance\@tempskipa-\prevdepth
             \advance\@tempskipa-\ht\nbsphinxbox
             \ifdim\@tempskipa>\z@
               \vskip \@tempskipa
             \fi}
    \unvbox\nbsphinxbox
    % if notebook starts with a \section, prevent it from adding extra space
    \@nobreaktrue\everypar{\@nobreakfalse\everypar{}}%
    % compensate the parskip which will get inserted by next paragraph
    \nobreak\vskip-\parskip
    % do not break here
    \nobreak
}% end of \nbsphinxstartnotebook

\newcommand{\nbsphinxstopnotebook}[1]{%
    \par
    % measure needed space
    \setbox\nbsphinxbox\vbox{{#1\par}}
    \nobreak % it updates page totals
    \dimen@\pagegoal
    \advance\dimen@-\pagetotal \advance\dimen@-\pagedepth
    \advance\dimen@-\ht\nbsphinxbox \advance\dimen@-\dp\nbsphinxbox
    \ifdim\dimen@<\z@
      % little space left
      \unvbox\nbsphinxbox
      \kern-.8\baselineskip
      \nobreak\vskip\z@\@plus1fil
      \penalty100
      \vskip\z@\@plus-1fil
      \kern.8\baselineskip
    \else
      \unvbox\nbsphinxbox
    \fi
}% end of \nbsphinxstopnotebook

% Ensure height of an included graphics fits in nbsphinxfancyoutput frame
\newdimen\nbsphinx@image@maxheight % set in nbsphinxfancyoutput environment
\newcommand*{\nbsphinxincludegraphics}[2][]{%
    \gdef\spx@includegraphics@options{#1}%
    \setbox\spx@image@box\hbox{\includegraphics[#1,draft]{#2}}%
    \in@false
    \ifdim \wd\spx@image@box>\linewidth
      \g@addto@macro\spx@includegraphics@options{,width=\linewidth}%
      \in@true
    \fi
    % no rotation, no need to worry about depth
    \ifdim \ht\spx@image@box>\nbsphinx@image@maxheight
      \g@addto@macro\spx@includegraphics@options{,height=\nbsphinx@image@maxheight}%
      \in@true
    \fi
    \ifin@
      \g@addto@macro\spx@includegraphics@options{,keepaspectratio}%
    \fi
    \setbox\spx@image@box\box\voidb@x % clear memory
    \expandafter\includegraphics\expandafter[\spx@includegraphics@options]{#2}%
}% end of "\MakeFrame"-safe variant of \sphinxincludegraphics
\makeatother

\makeatletter
\renewcommand*\sphinx@verbatim@nolig@list{\do\'\do\`}
\begingroup
\catcode`'=\active
\let\nbsphinx@noligs\@noligs
\g@addto@macro\nbsphinx@noligs{\let'\PYGZsq}
\endgroup
\makeatother
\renewcommand*\sphinxbreaksbeforeactivelist{\do\<\do\"\do\'}
\renewcommand*\sphinxbreaksafteractivelist{\do\.\do\,\do\:\do\;\do\?\do\!\do\/\do\>\do\-}
\makeatletter
\fvset{codes*=\sphinxbreaksattexescapedchars\do\^\^\let\@noligs\nbsphinx@noligs}
\makeatother



\title{Aquatic mammals}
\date{Mar 10, 2020}
\release{}
\author{}
\newcommand{\sphinxlogo}{\vbox{}}
\renewcommand{\releasename}{}
\makeindex
\begin{document}

\pagestyle{empty}
\sphinxmaketitle
\pagestyle{plain}
\sphinxtableofcontents
\pagestyle{normal}
\phantomsection\label{\detokenize{index::doc}}


Content generated from the OpenLearn Unit \sphinxhref{https://www.open.edu/openlearn/nature-environment/aquatic-mammals/content-section-0}{Aquatic mammals}.


\chapter{Contents:}
\label{\detokenize{index:contents}}

\section{Session 00}
\label{\detokenize{index:session-00}}

\subsection{1 The origins of aquatic mammals}
\label{\detokenize{content/session_00/Part_00_01:1-The-origins-of-aquatic-mammals}}\label{\detokenize{content/session_00/Part_00_01::doc}}
As an introduction to aquatic mammals, join Hermione Cockburn as she watches the bottlenose dolphins in the Moray Firth, Scotland in this video.

Video 1 Bottlenose dolphins in the Moray Firth









\sphinxstylestrong{HERMIONE COCKBURN:} \sphinxstyleemphasis{The Moray Firth is in the northeast of Scotland, and its beautiful and unspoiled coastline stretches for more than 800 km. It’s also home to one of just three populations of bottlenose dolphins that live around the UK. But unless they’re protected, these amazing creatures would struggle to survive. Today I’m in stunning Spey Bay to find out why.}; \sphinxstyleemphasis{Spey Bay lies on the south side of the Moray Firth, just to the east of where the River Spey flows into the sea, and it’s a
haven for the many birds and mammals that live along the shoreline.}; \sphinxstyleemphasis{There are 130 dolphins in the Moray Firth, but sadly research has shown that the population is declining by 6\% per year. These intelligent animals are increasingly vulnerable to threats like overfishing, getting tangled in nets, and pollution.}; \sphinxstyleemphasis{But if you don’t have an expensive camera or specialist training, don’t despair. Monitoring dolphins is something that anyone can do. Here at Chanonry Point, we’re a little bit
further along the Moray Firth, towards Inverness, and the dolphins come in so close, they’re practically right up to the beach. One of reasons for that is that there’s very strong tidal flow here, which acts a bit like a food conveyor belt, and the dolphins just hang around and wait for the fish to come past, which they can then ambush.};












\subsubsection{1.1 Introduction to aquatic mammals}
\label{\detokenize{content/session_00/Part_00_01:1.1-Introduction-to-aquatic-mammals}}
Aquatic mammals have clearly come a long way from their terrestrial ancestors and now they are superbly adapted to a fully aquatic life. The versatility of mammals will be obvious to you, but surely no environment has tested that versatility as much as the rivers and oceans of the world. As a first step, how many other aquatic mammals can you list?


\paragraph{Activity 1 Grouping aquatic mammals}
\label{\detokenize{content/session_00/Part_00_01:Activity-1-Grouping-aquatic-mammals}}
\sphinxstylestrong{Timing: Allow about 5 minutes}


\subparagraph{Question}
\label{\detokenize{content/session_00/Part_00_01:Question}}
Using your general knowledge of mammals, make a list of aquatic mammals that you can recall, dividing them into two groups: \sphinxstylestrong{aquatic} and \sphinxstylestrong{semi\sphinxhyphen{}aquatic}.


\subparagraph{Discussion}
\label{\detokenize{content/session_00/Part_00_01:Discussion}}












Aquatic mammals





Semi\sphinxhyphen{}aquatic mammals









whale





hippopotamus









seal





otter









sea lion





beaver









walrus





coypu









manatee





muskrat









dugong





water shrew













duck\sphinxhyphen{}billed platypus









Mammals are often considered to be a successful group, based on the fact that they are very widely distributed and flourish in seemingly hostile environments. Another useful measure of success might be numbers of species \textendash{} though in this case, the 4700 or so species of mammal are dwarfed by the known number of insect species in the world, which some estimate to be as many as 10 million. Other expressions of the success of mammals that are worth considering include the degree of physiological
sophistication of their systems (which can be difficult to assess) and the number of individuals of a particular species or broader group that exist in total.


\subsubsection{1.2 The origins of aquatic mammals}
\label{\detokenize{content/session_00/Part_00_01:1.2-The-origins-of-aquatic-mammals}}
Mammals are essentially a terrestrial group of animals: living mammals are descended from shrew\sphinxhyphen{}sized ancestors that evolved on land, and most mammals are still land\sphinxhyphen{}based animals.

\sphinxincludegraphics[width=512\sphinxpxdimen,height=341\sphinxpxdimen]{{s182_11_figure_2_megazostrodon}.jpg}

Figure 2 An early mammal, \sphinxstyleemphasis{Megazostrodon}

A shrew\sphinxhyphen{}like fossil of an early mammal, named \sphinxstyleemphasis{Megazostrodon}, is 200 million years old. Rather than shuffling along, with splayed\sphinxhyphen{}out limbs in the manner of many reptiles, this animal had limbs that were more erect and aligned under the body. Fossil evidence shows that the skulls of very early mammals have a distinctive lower jaw structure and sites on the skull for the attachment of chewing muscles. We can be confident that between 225 and 195 million years ago, mammal\sphinxhyphen{}like reptiles evolved
into true mammals, though for the next 100 million years or so these unobtrusive animals, none larger than a pet cat, co\sphinxhyphen{}existed with the dinosaurs.

For the first two\sphinxhyphen{}thirds of the evolutionary history of the mammals, from about 210 to 65 million years ago, dinosaurs dominated the plains and forests. But the dinosaurs conquered more than just the land: the seas and estuaries were home to other large aquatic reptiles, including ichthyosaurs and plesiosaurs. These animals disappeared at the same time as their terrestrial cousins \textendash{} and their demise opened up new habitats and new sources of food that the mammals were quick to exploit.


\paragraph{Question 1}
\label{\detokenize{content/session_00/Part_00_01:Question-1}}
\sphinxstylestrong{Timing: Allow about 5 minutes}


\subparagraph{Question}
\label{\detokenize{content/session_00/Part_00_01:id2}}
Using your general knowledge of mammals, try describing up to three biological features that mammals commonly possess.


\subparagraph{Discussion}
\label{\detokenize{content/session_00/Part_00_01:id3}}
Mammals are distinguished by:
\begin{itemize}
\item {} 
the production of milk

\item {} 
the possession of hair or similar structures, such as bristles

\item {} 
they have the ability to regulate their body temperature, usually keeping it above the temperature of their surroundings, an ability that they share with birds. They are often described as being warm\sphinxhyphen{}blooded.

\end{itemize}


\subsubsection{1.3 The pinnipeds, sirenians and cetaceans}
\label{\detokenize{content/session_00/Part_00_01:1.3-The-pinnipeds,-sirenians-and-cetaceans}}
Three major groups of mammals have independently adopted an aquatic way of life: the suborder Pinnipedia (part of the order Carnivora), the order Sirenia, and the order Cetacea. These groups will be the focus of your attention in this course. The smaller lineages that have become aquatic to some extent, such as the river and sea otters, won’t be covered in any detail.

You have just met the scientific names of some mammalian groups called orders. The class Mammalia is sub\sphinxhyphen{}divided into orders. Many scientific names have Latin or Greek roots and with a good dictionary (and maybe a web search), you may be able to unravel some of these words for yourself. For example, the Weddell seal which you will meet later in this course has the scientific name \sphinxstyleemphasis{Leptonychotes weddelli}. Scientific names are always printed in italics; if you write them by hand, you should
underline the name. In this instance, the Greek word \sphinxstyleemphasis{leptos} means ‘slender’ or ‘fine’, and \sphinxstyleemphasis{onyx} is a Greek word meaning ‘nail’ or ‘claw’. So the name of the genus \sphinxstyleemphasis{Leptonychotes} refers to the narrow claws on the flippers of the seal (actually, the hind flippers in this case). The species name \sphinxstyleemphasis{weddelli} is given in honour of Captain James Weddell, who described and illustrated the seal in his writings about an Antarctic voyage to catch seals in the 1820s. Weddell will forever be associated
with that seal, as specific names cannot be changed once formally given to a species.


\paragraph{Activity 2 Interpreting scientific names}
\label{\detokenize{content/session_00/Part_00_01:Activity-2-Interpreting-scientific-names}}
\sphinxstylestrong{Timing: Allow about 10 minutes}


\subparagraph{Question}
\label{\detokenize{content/session_00/Part_00_01:id4}}
You might like to try unraveling the derivation of a scientific name for yourself. Try the Pacific white\sphinxhyphen{}sided dolphin \sphinxstyleemphasis{Lagenorhynchus obliquidens}. Note down any thoughts or findings below.

(Hint: use a comprehensive dictionary to look up ‘lagena’, ‘rhyncho\sphinxhyphen{}’ and ‘oblique’.)


\subparagraph{Discussion}
\label{\detokenize{content/session_00/Part_00_01:id5}}
The generic name \sphinxstyleemphasis{Lagenorhynchus} should have been fairly easy to interpret, but the specific name of \sphinxstyleemphasis{obliquidens} might well have been more challenging. \sphinxstyleemphasis{Lagenorhynchus} is derived from the Greek ‘lagenos’ meaning ‘bottle’, and ‘rhynchus’ meaning ‘beak’. You might think that the bottlenose dolphin in Figure 1 would belong to this genus, but it does not \textendash{} a quirk of the naming process. The species name \sphinxstyleemphasis{obliquidens} means ‘oblique’ or ‘slanting’ tooth, referring to the fact that the teeth are
slightly curved.


\bigskip\hrule\bigskip


1.3.1 The suborder Pinnipedia The name of this suborder comes from a Latin word meaning ‘wing\sphinxhyphen{}footed’, which refers to the modification of limbs into flippers. Most pinnipeds move onto the land from time to time to rest and bask in the sun, and in the breeding season they gather, often in large numbers, on beaches or flat areas of sea\sphinxhyphen{}ice to give birth. There are three main families: the eared seals, which include fur seals and sea lions; the earless or true seals; and the walrus. Remember that
these mammals are carnivores; in view of their continuing links to land, they are sometimes aptly called ‘amphibious carnivores’.

The other two groups, the sirenians and cetaceans, are fully committed to a life in the water.


\bigskip\hrule\bigskip


1.3.2 The order Sirenia

\sphinxincludegraphics[width=512\sphinxpxdimen,height=164\sphinxpxdimen]{{s182_11_figure_3_composite}.jpg}

Figure 3 (a) Manatee, \sphinxstyleemphasis{Trichechus manatus}; (b) Drawing to show the possible appearance in life of the extinct Steller’s sea cow

There are four living members of the order Sirenia: three species of manatees and the dugong. These animals are descended from ancestors that also gave rise to the elephants, and they are the only herbivorous, completely aquatic mammals. The order is named after the Sirens of classical mythology, sea\sphinxhyphen{}nymphs who lured sailors to their deaths. Until relatively recently, there was a fifth species: Steller’s sea cow. This animal weighed around 10 metric tons, was 9\textendash{}10 m in length and lived mainly on
kelp. It had no teeth, though it did have rough plates on its palate. Hunting and the decline of the kelp beds caused a drastic fall in its numbers, and it holds the dubious record of the shortest period of time between discovery by western science (1741) and extinction (1768).

Manatees are amongst the earliest mammals to move into water. Fossils have been found that date from 50 million years ago. They are extremely well adapted to live in water and are found both in freshwater and the sea. One species lives in the Amazon. Smaller than the extinct Steller’s sea cow, manatees can reach 4 m in length.


\bigskip\hrule\bigskip


1.3.3 The order Cetacea The cetaceans get their name from the Greek word ‘ketos’ meaning ‘whale’. The order is divided into two suborders: the Odontoceti, or toothed whales, which include the sperm whale, killer whale, dolphins and porpoises; and the Mysticeti, or baleen whales, which include the blue whale, right whale and humpback whale, and are named for the brush\sphinxhyphen{}like plates that they use to filter krill from the water.

Video 2 Humpback whale filtering krill




\subsubsection{1.4 Moving to the water}
\label{\detokenize{content/session_00/Part_00_01:1.4-Moving-to-the-water}}
In moving to the water, aquatic mammals have had to survive, feed and reproduce using a set of biological characteristics that evolved in association with life on land. This course will explore how these characteristics have provided challenges \textendash{} as well as opportunities \textendash{} for mammals that spend some or all of their time in the water. You will examine some of the important differences between a life in air and a life in water, and learn how the bodies and behaviour of aquatic mammals have become
better suited to living and hunting in an aquatic environment. In Section 3, you will explore the so\sphinxhyphen{}called ‘diving response’, and some of the problems involved in finding out how the animals behave on their dives. Section 4 looks at what is now known about the evolution of the mammals that have made the most complete transition from the land to the water \textendash{} the whales. And finally, Section 5 will examine how human activity impacts on these ecosystems.


\subsection{2 Living in the water}
\label{\detokenize{content/session_00/Part_00_02:2-Living-in-the-water}}\label{\detokenize{content/session_00/Part_00_02::doc}}
This section will begin by examining the physiology of mammals, and looking at the differences between those who live on land and those in the water. You will see how the how the bodies and behaviour of aquatic mammals have adapted to tackle the challenges of an aquatic environment.


\subsubsection{2.1 Land versus water}
\label{\detokenize{content/session_00/Part_00_02:2.1-Land-versus-water}}
Mammals share a number of biological characteristics that mark them out as members of the class Mammalia. Many of these are adaptations to a life on land. For example:
\begin{itemize}
\item {} 
Mammals give birth to young at a relatively advanced stage of development and feed their young on milk.

\item {} 
Most mammals have hair, or fur, covering part or all of the body.

\item {} 
Mammals have a high metabolic rate and maintain a relatively high body temperature, irrespective of the temperature of their environment.

\item {} 
Most land mammals walk or run with a leg at each corner \textendash{} a limb pattern they share with most other land vertebrates, including reptiles (except snakes and legless lizards) and most amphibians.

\item {} 
Mammals breathe air and have a four\sphinxhyphen{}chambered heart, allowing them to split their circulation into a system for passing blood through the lungs and a system for transporting oxygen\sphinxhyphen{}rich blood around the body.

\end{itemize}

Almost all mammals can swim to some extent \textendash{} even bats \textendash{} and many make their way into the water from time to time for one reason or another.


\paragraph{Activity 3 Mammals in water}
\label{\detokenize{content/session_00/Part_00_02:Activity-3-Mammals-in-water}}
\sphinxstylestrong{Timing: Allow about 5 minutes}


\subparagraph{Question}
\label{\detokenize{content/session_00/Part_00_02:Question}}
\sphinxincludegraphics[width=880\sphinxpxdimen,height=986\sphinxpxdimen]{{s182_11_figure_5_elephant}.jpg}

Figure 4 A sequence of photos of an African elephant crossing a river

Figure 4 is a series of photos that show an African elephant crossing a river and snorkelling using its trunk. Study the photos and make notes about what might tempt any land\sphinxhyphen{}dwelling mammal into the water in the first place, and the main challenges it would face in the new environment.


\subparagraph{Answer}
\label{\detokenize{content/session_00/Part_00_02:Answer}}
The obvious attraction for many would be food, but other mammals might take to the water to escape from predators, pests or extremes of temperature. The elephant is a land mammal, but it enters the water to drink or bathe and as you have seen, it can swim across open water using its trunk as a snorkel.

The main challenges for a mammal moving into water would be: holding its breath long enough to find and catch its food; moving about in the strong current; and staying warm in the water. A good example would be the sea otter, which holds its breath while it uses webbed feet and a rudder\sphinxhyphen{}like tail to search for food in shallow coastal waters, its thick fur keeping it warm enough to live in Alaskan seas.


\paragraph{The evolution of elephants}
\label{\detokenize{content/session_00/Part_00_02:The-evolution-of-elephants}}
There are features of elephant anatomy that suggest that they might have adopted a semi\sphinxhyphen{}aquatic way of life at some point in their evolutionary history. The lungs of modern elephants are surrounded by loose connective tissue, not the bag\sphinxhyphen{}like ‘pleural cavity’ found in other land mammals, and this allows them to resist some of the damaging effects of water pressure. In addition, the kidneys of elephant foetuses possess features that are common in aquatic mammals but not land mammals. If the
theory is true, the ancestors of today’s elephants must have moved into the water and then returned to the land, while their sirenian cousins went on to become fully aquatic.

You might have thought of other examples of aquatic mammals, but they all commonly face the same challenges. They are the result of matching their mammalian characteristics against a number of differences in the physical properties of air and water, the most important of which are listed below.
\begin{itemize}
\item {} 
Water contains less oxygen than air does.

\item {} 
Water is denser and more viscous than air.

\item {} 
Water has a higher specific heat and a higher thermal conductivity than air (you will meet these terms later in Section 2.4).

\item {} 
Light is much more readily absorbed by water than by air, and visibility in water is often poor because suspended particles make it turbid.

\item {} 
Sound travels faster and further in water than it does in air.

\end{itemize}


\subsubsection{2.2 Breathing air}
\label{\detokenize{content/session_00/Part_00_02:2.2-Breathing-air}}
A swimming elephant can breathe by holding the end of its trunk out of the water, but if it tried to find its food under the surface, it would have to hold its breath. Neither the mammalian lung nor the skin can extract enough oxygen from water to sustain life, so fully aquatic mammals must come to the surface at intervals to breathe. All of them \textendash{} pinnipeds, sirenians and cetaceans \textendash{} are at risk of drowning if they are prevented from doing so for prolonged periods.

Lungs form 7\% of the body mass of a human being, but no more than 3\% of the body mass of most cetaceans \textendash{} so lung size alone cannot account for their diving abilities. Part of the secret is that aquatic mammals use their lungs more efficiently: they extract more oxygen from the air they take in, and change a higher proportion of the air within their lungs with every breath. For example, we take in about a litre of air with each breath, despite the fact that our lungs can hold over four litres; a
fin whale takes in 3000 times that amount in just under two seconds.

It is the ability to fill and empty their lungs so quickly that produces the ‘blow’ of the larger whales \textendash{} something so distinctive that it can be used to identify species, even individual whales, far out at sea. The animals pay a price for this efficiency, however. We can obtain extra oxygen by taking deeper breaths; a whale can only breathe more frequently. Whaling ships used to take full advantage of this fact, chasing whales until the exhausted animals were forced to spend more and more time
at the surface.

The need to breathe air means that sirenians and cetaceans have to be able to sleep without drowning, as these animals never leave the water. The blowhole in cetaceans seems to be under voluntary control, which suggests that these animals have to be ‘awake’ to breathe. From time to time, whales and dolphins rest at the surface of the water or close to it, and animals in social groups spend periods swimming slowly side by side (‘echelon swimming’). This paired swimming mimics the behaviour of a
calf with its mother and is probably associated with rest. It is likely that at these times the animals use a form of ‘cat\sphinxhyphen{}napping’, in which one half of the brain rests while the other remains active.


\subsubsection{2.3 Moving about}
\label{\detokenize{content/session_00/Part_00_02:2.3-Moving-about}}
Water is more viscous than air, so it can take more effort to move through water (try running in a swimming pool). Friction between the body and the water causes turbulence, which holds a swimmer back, and the faster the swimmer tries to move, the greater the turbulence. One way of avoiding the problem is to leave the water for short periods and travel through air, and some of the smaller pinnipeds and cetaceans resort to ‘porpoising’ \textendash{} leaping from the water for short periods when they need to
move as quickly as possible. But to catch their food, these animals have to be swift and agile in the water.

\sphinxincludegraphics[width=512\sphinxpxdimen,height=341\sphinxpxdimen]{{s182_11_figure_6_porpoising}.jpg}

Figure 5 Two dolphins porpoising

As pinnipeds breed on land (or ice), they have to be able to move around there as well. They have four flippers arranged in the typical mammalian limb pattern: two at the front and two at the back. In true seals, the hind flippers project behind the body and provide extra propulsion in association with a sideways swimming motion. The shorter front flippers are used for steering in the water and for pulling the body along on land.

\sphinxincludegraphics[width=512\sphinxpxdimen,height=192\sphinxpxdimen]{{s182_11_figure_6_composite}.jpg}

Figure 6 (a) Sea lions, \sphinxstyleemphasis{Zalophus wollebaeki}; (b) Fur seal, \sphinxstyleemphasis{Arctocephalus galapagoensis}

Fur seals and sea lions use longer front flippers to ‘fly’ through the water. Their hind flippers are mainly for steering, and most of the power comes from the body muscles and spine. On land, however, the hind flippers can be brought forward beneath the body, allowing the animal to lift itself up and ‘walk’.

Sirenians and cetaceans are fully aquatic and never leave the water. They swim in a rather different manner, using an up and down motion of a tail with horizontal flukes. The front fins help with steering and keep the animal stable in the water. This vertical swimming motion is another legacy of the animals’ terrestrial ancestry: imagine how the spine of a sprinting cheetah (or a galloping horse) flexes to allow the animal to increase its stride length.

The muscles of aquatic mammals are no stronger per unit of mass than the muscles of terrestrial mammals, so a dolphin’s top swimming speed of around 50 km h−1 is not the result of muscle strength alone. Most aquatic mammals are streamlined to some extent to help them slip through the water: having a smooth, cigar\sphinxhyphen{}shaped body with few external projections reduces water resistance and the drag created by turbulence. Streamlining is so important for aquatic animals that many fish, aquatic reptiles
such as the extinct ichthyosaur, and aquatic mammals share a common body shape \textendash{} an example of convergent evolution (see Figure 7). The streamlining of cetaceans is made even more effective by a special property of their skin. The outer layer is soft and spongy, and reduces turbulence at the skin\textendash{}water interface by expanding and contracting in response to the flow of water.

\sphinxincludegraphics[width=511\sphinxpxdimen,height=506\sphinxpxdimen]{{s182_11_figure_7}.jpg}

Figure 7 The streamlined bodies of (a) a shark, (b) an ichthyosaur (an extinct marine reptile) and (c) a dolphin

These adaptations are so effective that it is now believed that cetacean swimming is almost as cost\sphinxhyphen{}effective (in terms of metabolic energy expended for distance travelled) as the running of land animals. Swimming at the surface is less efficient, however, as is the swimming of semi\sphinxhyphen{}aquatic mammals such as otters. It appears that economic running and swimming are difficult to combine in the same animal.


\subsubsection{2.4 Staying warm …}
\label{\detokenize{content/session_00/Part_00_02:2.4-Staying-warm-_u2026}}
Water has a higher specific heat than air: that is, it takes more energy to raise the temperature of a particular volume of water than it does to raise the temperature of the same volume of air by the same amount. One consequence of this is that the temperature of surface water tends to fluctuate within a smaller range than air temperature: from −1.9 °C (the freezing temperature of seawater) near the poles to an extreme of about 30 °C in the tropics. But deep water is always cold, whatever the
location in the ocean; below about 1000 m, the temperature is uniformly cold, about 3\sphinxhyphen{}4 °C. The risks of heat loss from an aquatic mammal are all the greater because water has a relatively high thermal conductivity, so a warmer object loses heat fairly quickly when immersed. Think of thermal conductivity simply as a measure of how readily heat flows from a particular material. Marine mammals, like many of the larger land mammals, maintain their body temperature somewhere between 35 °C and 38 °C,
so they all face the problem of staying warm in a colder environment.



This section explained the term ‘thermal conductivity’ above as ‘a measure of how readily heat flows from a particular material’. In a course of this level and duration, there will not be a detailed explanation of how it is measured, or its actual values and units. You may like to do your own further research to learn more about this area.



To put this into context, consider the Weddell seal of Antarctica. It maintains an internal temperature of 37 °C, despite water temperatures of around 0 °C and air temperatures of 0 °C to −50 °C (with a wind chill down to −100 °C or so). The newborn pup is protected from the cold by a dense layer of fur called ‘lanugo’. In the first month or so of life, as it prepares to go to sea, the pup replaces its lanugo with the shorter, sparser hair of the adult seal and develops a thick layer of blubber.

\sphinxincludegraphics[width=512\sphinxpxdimen,height=471\sphinxpxdimen]{{s182_11_figure_9_weddell_mother_and_pup}.jpg}

Figure 8 Weddell seal and pup, sourced from \sphinxhref{https://www.coolantarctica.com/}{CoolAntarctica.com}




\paragraph{Question 2}
\label{\detokenize{content/session_00/Part_00_02:Question-2}}
\sphinxstylestrong{Timing: Allow about 5 minutes}


\subparagraph{Question}
\label{\detokenize{content/session_00/Part_00_02:id1}}
Can you think of a likely explanation for this change?


\subparagraph{Answer}
\label{\detokenize{content/session_00/Part_00_02:id2}}
Blubber provides better insulation in the water (and the smoother coat of the adult improves its streamlining). A fur coat traps a layer of warm air next to the body and is a fairly good insulator in air and water, but it has its limitations. It requires a lot of maintenance, and water pressure squeezes out the air if the animal travels too far below the surface. (It is the still air trapped by fur that provides the insulation; a trapped layer of water in fur would provide some insulation, but
not nearly as much as that offered by air.) Otters and fur seals make relatively modest dives, but many cetaceans and true seals, like the Weddell seal, hunt in deeper, colder waters.

Animals that dive to any great depth stay warm using blubber, a thick layer of fatty tissue that extends far beneath the skin. In air, blubber provides less insulation than an equal thickness of fur, but it retains much of its effectiveness in water and the extra bulk is less important in the more supportive medium. You’ll appreciate the fact that the relatively high density of the surrounding water helps ‘buoy up’ aquatic animals, such that diving mammals weigh relatively little in water.

\sphinxincludegraphics[width=780\sphinxpxdimen,height=311\sphinxpxdimen]{{s182_11_figure_8}.jpg}

Figure 9 A countercurrent heat exchange system. (a) Schematic representation. (b) Blood supply to the flipper of the dolphin, with a schematic cross\sphinxhyphen{}section of an artery and the surrounding veins to the left. Arterial blood is shown in red, venous blood in blue. Pink arrows denote heat flow; yellow arrows show direction of blood flow

Those parts of the animal not covered in blubber \textendash{} the limbs and parts of the head \textendash{} have another device for retaining heat: a countercurrent heat exchange system. In Figure 9(a), the outgoing (i.e. arterial) blood is shown to the left, on its way to the skin surface. But en route, such vessels run very close to a network of veins that are bringing cool, venous blood back into the body (shown to the right in Figure 9(a). Given the proximity of the (warm) outgoing and (cooler) incoming bloods,
heat (which would otherwise be lost through the skin) is taken up by the cooler returning blood and carried back inside the animal. The red arrows in Figure 9(a) show the direction of heat flow; there is comparatively little heat loss from the skin. If you think about it, the system depends on outgoing and incoming blood flowing in \sphinxstyleemphasis{opposite} directions, which is why it is called countercurrent heat exchange. If the two blood flows were in the \sphinxstyleemphasis{same} direction, there would cease to be any heat
transfer once the blood temperatures had become identical.

The system is shown to best effect in the flippers (and tails) of diving mammals, which have to be thin, and therefore free of insulating blubber, if they are to function effectively. Efficient heat exchange at such sites is essential if heat loss is to be avoided. Figure 9(b) shows the close juxtaposition of the arterial and venous bloods, with the central artery and surrounding veins in cross\sphinxhyphen{}section.

Fur and blubber are such effective insulators that some aquatic mammals can find themselves with another problem: losing excess heat after exercise, especially in sunny conditions. Blubber is full of small blood vessels which dilate to bring warm blood close to the skin, allowing a certain amount of active temperature regulation. The countercurrent system has its own control mechanism. A rise in blood pressure with activity causes the arterial walls to expand, which closes the network of veins
bringing cool blood back inside. The warm blood then passes straight through to the skin, excess heat is lost to the external environment, and the cool blood has to find another (non\sphinxhyphen{}countercurrent) route back to the heart.


\subsubsection{2.5 … and getting larger}
\label{\detokenize{content/session_00/Part_00_02:2.5-_u2026-and-getting-larger}}
Size offers a number of biological advantages, including some protection from predation, but it can also help in reducing heat loss. A large mammal has a large body mass (which generates heat) relative to its surface area (which loses heat). But there will be constraints on any evolutionary trend of increasing body mass in some aquatic mammals. Seals, sea lions and toothed whales have to remain agile enough to chase their prey, and the pinnipeds as a whole have to be able to move about on land,
but they are all still fairly large animals.

Manatees live in relatively warm waters. Their metabolic rate is comparatively low and they normally expend relatively little energy on keeping their body temperature constant. Their large size (up to 590 kilos and 4 m in length) helps conserve body heat but they are very vulnerable to cold water, below about 15 °C.

\sphinxincludegraphics[width=512\sphinxpxdimen,height=384\sphinxpxdimen]{{s182_11_figure_9_manatee}.jpg}

Figure 10 Manatee

The baleen whales, however, feed by filtering fish or invertebrates as they cruise the colder oceans of the world. As their prey is not chased, and their bodies are well supported by the density of the water, over the many millions of years of their evolution they have grown to immense sizes. The blue whale is the largest animal that has ever lived, a point that the following images will demonstrate.

\sphinxincludegraphics[width=512\sphinxpxdimen,height=298\sphinxpxdimen]{{s182_11_figure_10_african_elephant}.jpg}

Figure 11 The african elephant is the largest terrestrial mammal, at around 6 tonnes

\sphinxincludegraphics[width=512\sphinxpxdimen,height=341\sphinxpxdimen]{{s182_11_figure_11_blue_whale}.jpg}

Figure 12 The blue whale, at up to 150 tonnes, is the largest animal that has ever existed

\sphinxincludegraphics[width=880\sphinxpxdimen,height=480\sphinxpxdimen]{{s182_11_figure_11b_blue_whale_size_comparison}.jpg}

Figure 13 This comparative image then demonstrates the immense size of the blue whale, when shown to scale against the elephant, and a variety of other mammals


\subsubsection{2.6 Senses and communication …}
\label{\detokenize{content/session_00/Part_00_02:2.6-Senses-and-communication-_u2026}}
Light travels only a few hundred metres through clear water, so sight is of limited use to animals hunting in deeper waters.

Seals use their eyes and sensory whiskers to search for food in relatively shallow water. Elephant seals feed at depths of 300\textendash{}700 m and have eyes that adapt to poor light faster than those of any mammal tested; some six minutes compared with 20 minutes for the human eye. Despite this capacity, vision is probably of little use at the bottom of their feeding range and they must rely on sensitive whiskers and hearing. In fact, the whiskers of seals and sea lions are so important that a special
blood supply at their bases keeps them warm and flexible even in the deepest, coldest waters.

Eyesight is certainly of some use in making sense of the immediate environment, and sirenians need little else to find their food in shallow, clear waters. But when hunting at depth, as many toothed whales do, communication using sound is generally a better option. Sound travels five times faster in water than it does in air, and the toothed whales use a range of high\sphinxhyphen{}frequency squeaks, whistles and moans to communicate with each other and coordinate social behaviour, such as the coordinated
driving of fish by a pod of bottlenose dolphins, which you will find more about in Section 2.7.

You’ll also be aware that dolphins (and many other toothed whales) use a sophisticated echolocation system \textendash{} ‘seeing with sound’ \textendash{} to make sense of their environment and detect prey.

\sphinxincludegraphics[width=677\sphinxpxdimen,height=562\sphinxpxdimen]{{s182_11_figure_12}.jpg}

Figure 14 The generation and reception of sound in the head of a dolphin. The melon acts as an acoustic lens, focusing the clicks, while the oil\sphinxhyphen{}filled cavity of the lower jaw acts as an acoustic pipe, conducting sound to the ear

Dolphins produce a range of high\sphinxhyphen{}frequency clicks using air passages around the blow\sphinxhyphen{}hole (you can find many videos of dolphin clicks on youtube. This is one example of a \sphinxhref{https://www.youtube.com/watch?v=d6798extvoU}{video about how dolphin sounds work, with example audio}). These clicks are focused by the front of the skull, which is shaped like a satellite dish, and pass out into the water through a waxy structure, the ‘melon’, which acts as an acoustic lens. In this way, the animal can
focus and direct the stream of sound. The returning pulses are picked up by oil\sphinxhyphen{}filled cavities in the lower jaw and pass through to the inner ear, which is enclosed in a bone casing called the auditory bulla (Figure 14). This bone casing itself is surrounded by a mucus foam to isolate it from other vibrations. The whole system is sensitive enough to allow the animals to determine the size, shape and distance \textendash{} even the internal structure \textendash{} of objects as much as 800 m away. It is also powerful
enough to allow them to stun, or even kill, their prey with a well directed burst of sound.

As well as travelling faster in water, sound also travels further and with less distortion than in air. The other suborder of cetaceans \textendash{} the baleen whales \textendash{} utilise this property in their use of low\sphinxhyphen{}frequency calls produced from the vocal cords to communicate with each other across the oceans. The song of the humpback whale is perhaps the best\sphinxhyphen{}known example. The song differs from population to population, and individual to individual, and it develops over time as the whales imitate each other
and introduce variations. The whales also exploit a special property of the oceans to make themselves heard over great distances. At a depth of around 1000 m, there is a band of water in which temperature and pressure combine to reduce the speed of sound to a minimum. The low\sphinxhyphen{}frequency song of the whales can spread out through this ‘sound channel’ for distances of up to 1000 km, with little loss of energy to the water surrounding the channel.

Recent research on harbour seals and killer whales in the northeastern Pacific illustrates the sophistication of the sound world of marine mammals. The seals largely ignore a resident population of fish\sphinxhyphen{}eating killer whales, which use frequent echolocation clicks and communication calls. From time to time, however, the area is visited by transient pods of mammal\sphinxhyphen{}eating killer whales. These animals use few vocalisations, presumably to avoid alerting the seals, but the seals have learned to
recognise the sounds that they do make and take evasive action.


\paragraph{Activity 4 Whale vocalisations}
\label{\detokenize{content/session_00/Part_00_02:Activity-4-Whale-vocalisations}}
\sphinxstylestrong{Timing: Allow about 10 minutes}


\subparagraph{Question}
\label{\detokenize{content/session_00/Part_00_02:id3}}
To round off this section on senses and communication, the activity introduces you to some whale vocalisations and their possible function, or functions.

Now listen to the clicks produced by sperm whales at depth. The clicks are used for echolocation but might also be loud enough to ‘stun’ squid at close range. The clicks also have a social function. In the background, you can hear the high\sphinxhyphen{}pitched sounds of dolphins:

Audio 1 Sperm whale clicks



Humpback whales produce a wide range of sounds that are very complex and often produced for long sessions of several hours. Listen to a short clip and marvel at the structure of the ‘song’:

Audio 2 Humpback whale song



Hal Whitehead from the Department of Biology at Dalhousie University in Nova Scotia has a special interest in the social structure and cultural life of sperm whales, as he told the BBC’s Julian Hector:

Audio 3 Hal Whitehead interview









\sphinxstylestrong{HAL WHITEHEAD:} \sphinxstyleemphasis{The best evidence is\textendash{} for sperm whales and for most whales and dolphins, comes from their sounds. Sounds are very important in the ocean, they travel very well, much better than light or smell or anything like that, and we can record the sounds much more easily than we can see the animals. And so we’ve been able to listen to the sounds that the different whales and groups of whales are making, and to chart those patterns, and when we do this, we find patterns which don’t make
sense, just from the basis of genes controlling these things. So, it’s clear that the whales are learning their vocabulary, the sounds they make, from each other. Off the Galápagos, where I’ve done a lot of work, there are two very distinct clans of sperm whales, and one make patterns of clicks like ‘Click click click click’ or ‘click click click click click’. We call them the ‘regular clan’. And then there’s another clan who live in the same waters, who go ‘Click click click, click’ or ‘Click
click click click, click.’ We call them the ‘plus one’ clan. So they have these different ways of vocalizing, but there’s much more to it than that. The different social units form groups, they get together and work together for periods of days. They only do this with members of their own clan. The different clans have distinctive ways of moving around the islands. They have different feeding success. In normal years, when the waters are quite cool off the Galápagos, then the regular clan does
better. But when they warm up, you get El Niño coming in, then it’s reversed, and everyone does badly, but now it’s the plus one clan who are doing better. So it seems that these clans, which we \textendash{} and probably the whales themselves \textendash{} identify primarily through these sounds, their culture is a very important part of what they do, how they are and how they interact with the rest of the world.};

\sphinxstylestrong{JULIAN HECTOR:} \sphinxstyleemphasis{If they’re relying a lot on information being handed from one individual to another, and they can even share information from one distant group to another, depending what the year is, what impact does hunting on those populations have, if those groups are disrupted?};

\sphinxstylestrong{HAL WHITEHEAD:} \sphinxstyleemphasis{The first idea is that there may be particular animals, and probably in particular older females, who have very important sets of knowledge, and as things in the ocean are varying over long timescales, the older animals’ knowledge becomes particularly important, because they’ve been through a lot of the variations which happen very rarely. So, especially if you lose a lot of the older animals in your society, it can really affect the ones who are left. And we have indications
of that in the Eastern Pacific, where the whaling between about the 1950s and the early 1980s was particularly intense, and there it’s likely there aren’t that many very older animals, and we find less success in raising calves than say, in the Western Atlantic where there wasn’t that modern whaling. So that’s one level of it, and another level is that, if you get a population down to very small size, then not only do you lose genes, genetic diversity, but you lose cultural diversity, so, some
of the knowledge that has been important for that species may be lost, just because there’s so few animals left, and those that are left have either forgotten or not learned that particular knowledge. So for instance, the right whales in the North Atlantic and North Pacific, which have been brought down to probably tens of animals, seem not to use many of the areas that traditionally they did use, and so in some ways perhaps they’ve lost important cultural knowledge of how to use their
environment.};

\sphinxstylestrong{JULIAN HECTOR:} \sphinxstyleemphasis{So in your view, Hal, is this an argument to ban whaling or to limit whaling?};

\sphinxstylestrong{HAL WHITEHEAD:} \sphinxstyleemphasis{I think it’s an argument to ban whaling, as well as possibly to limit whaling in some cases. I think the third argument which I didn’t make then is the idea that, the culture is a good marker of these other cognitive traits, that are characteristic obviously of humans, but of a very few other animals which we’re beginning to think maybe we should give special rights to. So I think we need to think about that \textendash{} these are difficult issues, but if we’re going to give special
rights to chimpanzees, we’re not going to chop them up for meat, then I think the same arguments hold for whales, and we shouldn’t be doing the same for them.};











You can listen to recordings of lots of other whale species online (on \sphinxhref{https://en.wikipedia.org/wiki/Whale\_vocalization\#Media}{Wikipedia} for example).


\subsubsection{2.7 … and becoming more intelligent}
\label{\detokenize{content/session_00/Part_00_02:2.7-_u2026-and-becoming-more-intelligent}}
Intelligence is a useful commodity: it can help an animal to make sense of its environment and cope with the demands of social behaviour such as courtship and competition. Hunters tend to be relatively intelligent, and otters, pinnipeds and cetaceans, for example, share a playful curiosity that is characteristic of animals that catch other animals in order to survive. Some especially extravagant claims have been made for the intelligence of the toothed whales, largely because these animals use
communication and cooperation to maximise the efficiency of their hunting behaviour.


\paragraph{Activity 5 Intelligent behaviour?}
\label{\detokenize{content/session_00/Part_00_02:Activity-5-Intelligent-behaviour?}}
\sphinxstylestrong{Timing: Allow about 10 minutes}


\subparagraph{Question}
\label{\detokenize{content/session_00/Part_00_02:id4}}
Watch the video clip, which shows the hunting behaviour of bottlenose dolphins. Note down any aspects of the animals’ behaviour that strike you as intelligent.

Before you watch, it might be an idea to spend a few minutes thinking about what ‘intelligent’ means. But be warned, it’s by no means an easy concept to pin down.

Video 3 Hunting behaviour of dolphins




\subparagraph{Answer}
\label{\detokenize{content/session_00/Part_00_02:id5}}
You may have noted the following points:
\begin{itemize}
\item {} 
Strategy and tactics: the dolphins identify a suitable target (a shoal of fish) and plan ahead (edging the shoal close to the bank or the surface of the water).

\item {} 
Teamwork and cooperation: working together to achieve a common goal.

\item {} 
The use of communication to coordinate behaviour.

\item {} 
The adoption of specific roles within the group (the ‘spy\sphinxhyphen{}hopper’, for example).

\item {} 
The use of judgement to select the right option at the right time.

\end{itemize}

Some of the words you probably noted down certainly imply intelligence \textendash{} ‘planning’, ‘communication’, ‘judgement’ \textendash{} but it is difficult to define or measure intelligence in ourselves, let alone other animals. The behaviour of the dolphins certainly seems as sophisticated as the hunting strategies of lions or African hunting dogs, for example, but is there any evidence that they are more intelligent than that?

Well, cetaceans have large brains relative to body size, with folds in the part of the brain responsible for complex functions \textendash{} the cerebral cortex \textendash{} that are reminiscent of those in primates. They use sound in a number of sophisticated ways and engage in subtle social interactions that may be mediated by a simple form of ‘language’. Dolphins can be trained to respond to commands and perform tricks \textendash{} and they can develop and vary these routines and imitate each other’s behaviour. On the other
hand, the large brain and the folding of the cerebral cortex may be simply a function of the size of these animals, or a consequence of the need to produce and process such complex sounds. And many other animals can be trained to perform tricks, including pigeons and other species not generally regarded as particularly bright.

One behaviour that has caused some people to question the intelligence of cetaceans is a tendency to strand themselves in large numbers on beaches and, even more puzzling, to strand themselves again if they are helped back into the water. The reasons for this phenomenon are poorly understood. It could be a response to distress calls, a form of mass panic, or the result of damage to the echolocation system caused by disease, parasites or pollution. As with so many aspects of cetacean behaviour,
the honest answer is that we just don’t know.

One difficulty in interpreting cetacean behaviour may be our tendency to describe and discuss intelligence in terms of the way we, as humans, interact with each other and our surroundings. The arrogance of attempting to judge the intelligence of other species by our own standards was used to comic effect by Douglas Adams in \sphinxstyleemphasis{The Hitchhiker’s Guide to the Galaxy}.


\begin{quote}

It is an important and popular fact that things are not always what they seem. For instance, on the planet Earth, man had always assumed that he was more intelligent than the dolphins because he had achieved so much \sphinxhyphen{} the wheel, New York, wars and so on \sphinxhyphen{} whilst all the dolphins had ever done was muck about in the water having a good time. But conversely, the dolphins had always believed that they were far more intelligent than man \sphinxhyphen{} for precisely the same reasons. (Adams, 1979, p.132)
\end{quote}


\subsection{3 The ‘diving response’}
\label{\detokenize{content/session_00/Part_00_03:3-The-_u2018diving-response_u2019}}\label{\detokenize{content/session_00/Part_00_03::doc}}
All the aquatic mammals’ adaptations to life in the water \textendash{} breathing, moving, staying warm and making sense of the environment \textendash{} come together in their diving behaviour, and the diving abilities of marine mammals are truly awe\sphinxhyphen{}inspiring. This section will explore these adaptations.


\subsubsection{3.1 Features of the diving response}
\label{\detokenize{content/session_00/Part_00_03:3.1-Features-of-the-diving-response}}
The elephant seal (Figure 15(a)) makes repetitive, long\sphinxhyphen{}duration dives and some 80\textendash{}95\% of its time at sea is spent submerged. Its dives are of about 20 minutes duration on average, and the intervals at the surface are seldom more than about two minutes. This pattern is repeated almost continuously for between two and eight months. Such abilities are especially useful for those seals that often travel long distances under ice from one breathing hole to another \textendash{} the Arctic\sphinxhyphen{}dwelling ringed seal is
one such example and its Antarctic counterpart, the Weddell seal is another. The sperm whale (Figure 15(b)) routinely swims down to depths in excess of 1000 m to hunt giant squid and can stay submerged for over an hour.

Video 4 Elephant seals









\sphinxstylestrong{DAVID ROBINSON:} \sphinxstyleemphasis{It’s late September and the windswept coast on the Peninsular Valdes in Patagonia is dotted with southern elephant seals. A male, readily identified by his large size and prominent proboscis, has hauled out onto the beach and will compete with other males to establish a harem. While ashore, females will give birth, but won’t feed until the pup is weaned. A male is eight to ten times the mass of a female and will spend 80\% or more of his life at sea. Elephant seals can dive
deep into the ocean in search of food and the deepest recorded dive is 2388 m.};











\sphinxincludegraphics[width=512\sphinxpxdimen,height=141\sphinxpxdimen]{{s182_11_figure_13_composite}.jpg}

Figure 15 (a) Southern elephant seal; (b) Sperm whale

The length of time a mammal can spend under water depends on the amount of oxygen it takes on board before a dive, as well as its ability to slow the rate at which it uses that oxygen once the dive is under way. It was not until the 1930s and 1940s that biologists began to understand the diving abilities of marine mammals. Work on captive seals making simulated dives in the laboratory \textendash{} mainly Weddell seals and elephant seals \textendash{} identified a number of physiological mechanisms that became known as
the ‘diving response’, which is detailed below. Subsequent research involving monitoring animals in their natural environment has shown that the full diving response is necessary only \sphinxstyleemphasis{in extremis}, when the animal is pushed close to its physiological limits.

The main elements of the diving response are as follows:
\begin{itemize}
\item {} 
The animal stops breathing (technically known as ‘apnoea’).

\item {} 
The heart rate slows very markedly \textendash{} a condition termed ‘bradycardia’. Many seals can reduce their heart rate to about four beats per minute.

\item {} 
The blood pumped by the heart is diverted away from peripheral tissues and muscles to the oxygen\sphinxhyphen{}dependent heart, brain and other essential organs. In essence, the animal turns itself into a ‘heart\sphinxhyphen{}lung\sphinxhyphen{}brain’ machine.

\item {} 
The overall metabolic rate of the diving animal falls as some tolerant tissues \textendash{} the gut and kidneys, for example \textendash{} are starved of oxygen. The metabolic rate also falls as the temperature of peripheral tissues \textendash{} the flippers or flukes, for example \textendash{} moves closer to that of the surrounding water. (In general, the warmer a tissue, the greater its metabolic rate and subsequent demand for oxygen.) The countercurrent heat exchange ensures that the blood supply to these peripheral tissues is
maintained even as tissue temperature falls (Figure 9(b)).

\item {} 
The animal swims in a slow and sustained manner, gliding along to conserve oxygen.

\end{itemize}

These adaptations allow the animal to make efficient use of the oxygen carried on the dive. Two other adaptations allow the animals to increase the amount of oxygen carried, providing what some biologists have called a ‘physiological scuba tank’:
\begin{itemize}
\item {} 
Oxygen is stored by two proteins in the body: haemoglobin in the red blood cells, and myoglobin in the muscles. Diving mammals have exceptional quantities of both substances.

\item {} 
In some diving mammals, muscles in the spleen contract to squeeze out more oxygen\sphinxhyphen{}holding red blood cells when the animal dives. The spleen is also larger, allowing it to hold and release more red blood cells.

\end{itemize}

Perhaps surprisingly, the main elements of the diving response are not confined to aquatic mammals. A similar response is found in all mammals and most other air\sphinxhyphen{}breathing vertebrates, including reptiles and birds. The observation that the diving response seems to be ‘hard wired’ into all mammals has led to some debate about its wider purpose. It may be that it has a role in other situations in which the body has to cope with reduced levels of oxygen.


\paragraph{Question 3}
\label{\detokenize{content/session_00/Part_00_03:Question-3}}
\sphinxstylestrong{Timing: Allow about 5 minutes}


\subparagraph{Question}
\label{\detokenize{content/session_00/Part_00_03:Question}}
Can you think of any circumstances in which the ability to withstand oxygen starvation might be of use to a terrestrial animal?


\subparagraph{Answer}
\label{\detokenize{content/session_00/Part_00_03:Answer}}
There are a number of possible situations, including vigorous exercise or oxygen starvation during birth. Of course, we are also mammals and we have our own diving response. Freedivers swim as deep as they can without the aid of scuba tanks. Most human beings can endure apnoea for one to two minutes and dive to a depth of 10\textendash{}20 m. But it is possible to train to dive for longer periods. The current world freediving record holder, Herbert Nitsch, dived to 253.2 m in 2012. It is possible, with
training, for a human to hold their breath for long periods, for example the 22 minutes that Stig Severinsen spent immersed in a tank in 2012. Freedivers can also reduce their heart rates to an amazing six beats per minute, which is not far away from the minimum of a diving seal.


\subsubsection{3.2 Natural dives}
\label{\detokenize{content/session_00/Part_00_03:3.2-Natural-dives}}
The physiology of the diving response can be studied in the laboratory, but investigating the behaviour of a diving mammal in its natural environment can be more of a problem. However, modern physiological techniques have made it possible to record continuously physiological variables (such as heart rate) and information on depth and position during the spontaneous dives in the wild that are part of the animal’s normal behaviour. Most such findings show that the majority of an animal’s dives
don’t approach the ‘Olympic record’ limits that represent maximum effort. For example, most Weddell seal dives were shorter than 20 minutes, and there is little evidence of profound adjustments in the heart and circulatory system during this time, or of the presence of lactic acid in the blood of these animals as they surface. Lactic acid is an important indicator of the onset of anaerobic metabolism (i.e. metabolism that doesn’t use oxygen). The production of lactic acid is a relatively
inefficient way of releasing energy by the breakdown of simple sugars obtained from the diet. (In humans, lactic acid appears in the bloodstream during intense activity such as sprinting, when the supply of oxygen to the muscle cells isn’t sufficient to allow the oxidation of sugars in the ‘conventional’ way.)


\paragraph{Question 4}
\label{\detokenize{content/session_00/Part_00_03:Question-4}}
\sphinxstylestrong{Timing: Allow about 5 minutes}


\subparagraph{Question}
\label{\detokenize{content/session_00/Part_00_03:id1}}
What do these observations suggest about the physiological events that accompany natural dives?


\subparagraph{Answer}
\label{\detokenize{content/session_00/Part_00_03:id2}}
They suggest that a full\sphinxhyphen{}scale diving response is not typical of these shorter dives. If tissues were starved of oxygen, substantial levels of lactic acid would probably be evident in the blood, as anaerobic metabolism became established. The implication is that in shorter dives, the amount of oxygen carried by the Weddell seal (mainly in the blood, as we’ve seen) is sufficient to tide the animal over, without a sustained or profound diving response. But with natural dives of 45 minutes or so,
anaerobic metabolism becomes much more significant and the selective re\sphinxhyphen{}routing of blood (together with bradycardia) becomes much more important.

What about hunting behaviour? To continue the same example, the Weddell seal spends most of its time hunting cod and other fish out of sight beneath the Antarctic ice.

Marine biologist Terrie Williams and her colleagues at the University of California have studied the hunting behaviour of Weddell seals by attaching recording equipment to animals captured near their feeding holes. The seals are fitted with video and audio recorders, plus monitors that note the animals’ depth, speed, compass bearing and flipper strokes. The seals are then released back into the sea. The information that the scientists receive allows them to construct a three\sphinxhyphen{}dimensional map of
the dive that is tied to specific pieces of behaviour recorded by the video camera. The seals hunt in low levels of light and have to find, stalk and catch their prey while holding their breath.

\sphinxincludegraphics[width=621\sphinxpxdimen,height=424\sphinxpxdimen]{{s182_11_figure_14}.jpg}

Figure 16 A three\sphinxhyphen{}dimensional map of the dive of a Weddell seal. The numbers on the plot indicate elapsed time in minutes since the start of the dive, and point C is where the seal made contact with the fish. Speed is indicated on the scale at the bottom right of the dive map. If you try and read off the depth of the seal using the vertical axis, you’ll note that it doesn’t correspond exactly; we have reproduced the figure accurately from the original publication but it shows the difficulty of
representing changes in three dimensions using only two dimensions!

Figure 16 shows a dive by an adult female Weddell seal. The dive lasted for 10 minutes 30 seconds, and the animal travelled a total of 760 m. About midway through the dive, at point C, the video footage revealed that the seal attacked a large Antarctic cod, approaching from behind and below its tail. Let’s look at what happened in more detail.

The seal started the dive with a few powerful strokes of its flippers. Water pressure would have quickly collapsed its lungs, making the seal denser than the surrounding water. The negatively buoyant animal could then have continued its descent with little effort. (The team have discovered that in pinnipeds and cetaceans, this effortless gliding minimises the effort expended by the animal while submerged.) The seal sank slowly to a depth of 51 m (marked by the right arrow), rose slightly to a
depth of 33 m (marked by the left arrow), and then started a new descent, maintaining roughly the same bearing and speed. This course was taking it in the general direction of the fish.

At point A, some 23 m from the fish, the seal used a few large flipper strokes to change direction and speed, accelerating away from the fish at an angle of 58 degrees to its previous course. It then used a looping turn (point B) to bring it back towards \textendash{} but beneath \textendash{} the fish. It accelerated again towards the target but slowed just before contact was made. After the attack had failed, the seal did not pursue the fish but continued its descent for a while, before turning left (point D) and
proceeding quickly back to the ice hole.


\paragraph{Question 5}
\label{\detokenize{content/session_00/Part_00_03:Question-5}}
\sphinxstylestrong{Timing: Allow about 5 minutes}


\subparagraph{Question}
\label{\detokenize{content/session_00/Part_00_03:id3}}
The data gathered can tell us a lot about \sphinxstyleemphasis{what} happened, but little about \sphinxstyleemphasis{why}. We are still left with the problem of interpreting the sequence. Try this for yourself. For example, why might the seal have moved \sphinxstyleemphasis{away} from the fish at point A?


\subparagraph{Answer}
\label{\detokenize{content/session_00/Part_00_03:id4}}
Here is one interpretation (and it is difficult, if not impossible, to prove). It’s possible that the initial descents and ascents on a constant bearing were a searching strategy. At point A, the seal spotted the cod a short distance away (remember the poor light). If it approached the fish head\sphinxhyphen{}on it would be seen quite quickly. The turns at points A and B allowed the seal to attack the fish from behind and below.

After accelerating out of the second turn, it pulled its flippers to its side and coasted towards its prey, which accounts for the deceleration recorded just before contact. The seal would be camouflaged against the darkness of the deep water and the fish would be silhouetted against the ice.


\paragraph{Optional activity}
\label{\detokenize{content/session_00/Part_00_03:Optional-activity}}
\sphinxstylestrong{Timing: Allow about 30 minutes}


\subparagraph{Question}
\label{\detokenize{content/session_00/Part_00_03:id5}}
If you find this topic interesting, you have the opportunity to find out more from the original scientific paper by Terrie Williams and her colleagues that reported these findings to the scientific world. The language used in the paper is sometimes technical and assumes some familiarity on the part of the reader with specialist terminology and the use of recording techniques. But you don’t need to understand it all to get something from the paper. It contains much more information than the
signficantly abbreviated description above, outlining the reasons the research work was undertaken, and providing a fuller account of the three brief encounters \textendash{} from 57.4 hours of recording! The broader significance of these findings is discussed, with an emphasis on what was new and unexpected about these observations, and the potential of the technique for revealing more about the hunting behaviour of diving mammals. Click the following link to view the paper. \sphinxhref{https://doi.org/10.1126/science.283.5404.993}{Link to paper on hunting
behaviour}


\subsection{4 The evolution of whales}
\label{\detokenize{content/session_00/Part_00_04:4-The-evolution-of-whales}}\label{\detokenize{content/session_00/Part_00_04::doc}}
Now that you have examined the challenges facing all aquatic mammals, it is interesting to consider the evolutionary pathway followed by them. There isn’t space in this short course to look at all the groups in detail. So, as a brief example, this section will explore what is known about the evolutionary history of the cetaceans.

\sphinxincludegraphics[width=512\sphinxpxdimen,height=191\sphinxpxdimen]{{s182_11_figure_15_composite}.jpg}

Figure 17 Humpback whales


\subsubsection{4.1 The rate of evolution}
\label{\detokenize{content/session_00/Part_00_04:4.1-The-rate-of-evolution}}
The cetaceans have travelled furthest from their terrestrial roots and made the fullest adaptation to life in the sea. Since mammals evolved on land, it has long seemed reasonable to suggest that the origin of whales must have involved an evolutionary transition from the land to the water. But how can we explain the fact that living whales are so well adapted to a life in the water, and so far removed in form and behaviour from most other mammals? There are two main problems. First, if evolution
proceeds through the build\sphinxhyphen{}up of numerous small changes, it is difficult to see how the wide gap between terrestrial mammals and whales could have been bridged \textendash{} even given the enormous amounts of time involved. Second, there is the problem of ‘intermediate forms’: that is, the nature of the animals that must have filled the gap.

In the 1930s, the American biologist George Gaylord Simpson noted something interesting about the apparent rate of evolutionary change in whales. He examined the fossils of some primitive whales, which had been dated back to the Eocene (about 50 million years ago), and compared the anatomy of these early proto\sphinxhyphen{}whales with the anatomy of their living descendants. Eldredge describes Simpson’s work:


\begin{quote}

One can measure the average rate of evolution for various anatomical features in the 50 million years it took to modify {[}primitive{]} Eocene whales into fully modern forms. Let us then take that measured rate of evolution within whales and calculate how long it would have taken for Eocene whales to evolve from terrestrial ancestors. Extrapolating back, it would have taken at least 100 million years (possibly even considerably more) for the transition from terrestrial ancestor to aquatic,
primitive whale descendant to have occurred \sphinxhyphen{} assuming, that is, that whales evolved from terrestrial ancestors at the same rate of evolution we see in the 50 million years that elapsed between Eocene and modern whales. (Eldredge, 1991, p. 168)
\end{quote}

This is a problem: 100 million years before the Eocene takes us back before the evolution of mammals as a group (let alone the first mammals to take to the water).


\paragraph{Question 6}
\label{\detokenize{content/session_00/Part_00_04:Question-6}}
\sphinxstylestrong{Timing: Allow about 5 minutes}


\subparagraph{Question}
\label{\detokenize{content/session_00/Part_00_04:Question}}
Can you identify a possible solution to this problem? (Hint: think about the assumption noted in the quotation.)


\subparagraph{Answer}
\label{\detokenize{content/session_00/Part_00_04:Answer}}
Perhaps the rate of evolution was \sphinxstyleemphasis{not} the same for the transition from terrestrial ancestor to primitive Eocene whale as it was for the later transition from Eocene whale to modern whales. Simpson suggested that the rate of evolution had \sphinxstyleemphasis{not} been the same \textendash{} that the large\sphinxhyphen{}scale transformations needed for a mammal to move from the land to water might have occurred more quickly than the subsequent ‘fine\sphinxhyphen{}tuning’ from primitive whale to modern whale. Some biologists see this as a general
pattern, in which periods of rapid evolutionary change in small, isolated populations give rise to major new groups of animals, and are followed by longer periods of relative stability in which the rate of change to the new design is fairly modest. Niles Eldredge and the late Stephen Jay Gould have coined the term ‘punctuated equilibria’ for this controversial theory.


\subsubsection{4.2 Intermediate forms}
\label{\detokenize{content/session_00/Part_00_04:4.2-Intermediate-forms}}
In essence, the argument about intermediate forms runs as follows. If whales evolved from a terrestrial ancestor through the accumulation of small differences over time, we should expect to find the fossils of a number of ‘missing links’, i.e. creatures with a mixture of terrestrial and aquatic characteristics. In fact, we might expect to find a succession of such animals, each a little bit more whale\sphinxhyphen{}like and a little bit less well adapted to life on land than its predecessor.

To make things more complicated, each of these intermediate forms must have been a fully working animal: it must have been able to breathe, to move about its environment, to feed itself and to reproduce. For a long time, biologists speculated about what these animals might have looked like. Surely at some point there must have been a creature that was at home on neither land nor sea, and so unable to compete with the animals already fully adapted to either habitat.

The problem of intermediate forms is by no means a new one. In \sphinxstyleemphasis{The Origin of Species}, published in 1859, Charles Darwin included a whole chapter on some of the difficulties facing his new theory. He started with the problem of intermediate forms:


\begin{quote}

Firstly, why, if species have descended from other species by insensibly fine gradations, do we not everywhere see innumerable transitional forms? (Darwin, 1859, p. 130)
\end{quote}

And a little later in the same chapter:


\begin{quote}

It has been asked by the opponents of such views as I hold, how, for instance, a land carnivorous animal could have been converted into one with aquatic habits; for how could the animal in its transitional state have subsisted? (Darwin, 1859, p. 136)
\end{quote}

Let’s examine Darwin’s questions one at a time. First, the scarcity of intermediate forms in the fossil record.


\paragraph{Question 7}
\label{\detokenize{content/session_00/Part_00_04:Question-7}}
\sphinxstylestrong{Timing: Allow about 5 minutes}


\subparagraph{Question}
\label{\detokenize{content/session_00/Part_00_04:id1}}
What does the theory of ‘punctuated equilibria’ suggest about the chances of finding intermediate forms in the fossil record?


\subparagraph{Answer}
\label{\detokenize{content/session_00/Part_00_04:id2}}
The theory suggests that we are more likely to find fossils of the animals that occupy the relatively stable periods (the ‘equilibria’) than we are to find the fossils of the animals that form part of the periods of rapid change, especially if that rapid change took place in small or isolated populations. Now let’s think about the nature of those animals.


\paragraph{Question 8}
\label{\detokenize{content/session_00/Part_00_04:Question-8}}
\sphinxstylestrong{Timing: Allow about 5 minutes}


\subparagraph{Question}
\label{\detokenize{content/session_00/Part_00_04:id3}}
Can you think of any modern animals that are at home on land and in the water?


\subparagraph{Answer}
\label{\detokenize{content/session_00/Part_00_04:id4}}
Otters are wonderful swimmers and can chase and catch fish, but they also move about well enough to hunt on land. Seals and sea lions are clearly most at home in the water, but they can still chase each other about on the breeding beaches. So it is possible to imagine an early ancestor of the whales leading a similarly amphibious existence. Studies of the chemical make\sphinxhyphen{}up of proteins (particularly antibodies) in present\sphinxhyphen{}day species have shown that modern whales are closely related to the
artiodactyls, a group of hoofed mammals that includes cows, sheep, deer and hippos. For some 100 years after the publication of \sphinxstyleemphasis{The Origin of Species}, all the known fossils of early cetaceans were either clearly terrestrial animals or primitive whales like those examined by Simpson. But recent finds have included the remains of creatures that might have had a more amphibious existence. These animals also lived in the Eocene (55 to 34 million years ago), in or around the warm, shallow waters of
the Tethys Sea, an ancient ocean that stretched from modern Spain to Indonesia.

\sphinxincludegraphics[width=780\sphinxpxdimen,height=889\sphinxpxdimen]{{s182_11_figure_16}.jpg}

Figure 18 Some representative mammals in the evolutionary history of modern whales. These are not drawn to scale (measurements refer to body length) and the dashed lines do not reflect direct descent. Shown here are imaginative reconstructions of the fossil species, suggesting how they might have looked in life

Figure 18 shows three representative animals in a postulated evolutionary sequence from an Eocene terrestrial ancestor to modern odontocetes and mysticetes.

\sphinxstyleemphasis{Pakicetus} is approximately 50 million years old. It lived by the side of shallow marine estuaries and was probably a wolf\sphinxhyphen{}sized ambush predator. It had long legs, an upright stance and may have spent much of its time running down small fish in the shallows.

\sphinxstyleemphasis{Ambulocetus} (‘walking whale’) is approximately 48 million years old, somewhere between \sphinxstyleemphasis{Pakicetus} and \sphinxstyleemphasis{Dorudon}. It was an amphibious animal with many of the characteristics we would expect in a transitional form. It could probably clamber about on land like a crocodile, but it was also a powerful swimmer, using its large feet and flexible spine like an otter. Intriguingly, it seems to have had a fluid\sphinxhyphen{}filled cavity linking its jaw and inner ear, similar to that found in modern dolphins
(Figure 14), and it may have listened for the vibrations of approaching prey by resting its jaw on the ground \textendash{} a strategy seen in today’s crocodiles.

\sphinxstyleemphasis{Dorudon} is somewhere between 40 and 36 million years old. This animal lived in warm, shallow seas, where it fed on small fish and molluscs. It was about 4.5 m long and, like modern cetaceans, fully committed to a life in the water.

The Eocene was followed by the Oligocene (34 to 24 million years ago). The Tethys Sea disappeared as India ploughed into Asia, and the early whales found themselves having to live in deeper, colder waters. It didn’t take them long, in evolutionary terms, to split into the two main lifestyles we see today. The toothed whales came to hunt fish and squid using high\sphinxhyphen{}frequency echolocation, while the baleen whales began to feed by filtering water, grew ever larger and sang their low\sphinxhyphen{}frequency songs
across the oceans of the world.


\subsection{5 Human impact}
\label{\detokenize{content/session_00/Part_00_05:5-Human-impact}}\label{\detokenize{content/session_00/Part_00_05::doc}}
In the introduction to this course, you considered the success of the mammals as a whole. You would probably also agree that the aquatic mammals are a successful group too. However, there is another mammal that has been successful in a different way and has challenged the success of some of the aquatic mammals.

There is only one species of human, but its capacity to occupy and exploit almost all ecosystems has brought conflict. The seals and whales have been hit hard by the impact of humans harvesting them for fur, food and oil. Whale populations have suffered huge losses. Just one illustration of this is a series of figures for the number of whales caught in just one small area of the Southern Ocean around the South Shetland and South Orkney Islands in a period of 19 years during the height of
Antarctic whaling. The figures come from a plaque originally on the site of a Norwegian whaling station on Signy Island.













Species





right





blue





fin





sei





humpback





sperm





Total









Number caught





78





61 366





48 023





1 796





6 742





18





118 159









Exploitation on this scale was never going to be sustainable. Just think about some of the reasons, before reading on.

Whales are large animals and have long life cycles and slow reproductive rates. Whalers generally hunted larger individuals, those that were most likely to be of breeding age.

\sphinxincludegraphics[width=512\sphinxpxdimen,height=302\sphinxpxdimen]{{s182_11_figure_17_humpback_breaching}.jpg}

Figure 19 A breaching humpback whale, \sphinxstyleemphasis{Megaptera novaeangliae}

Recently, as shipping lanes have become much more crowded, the risk of collisions between whales and ships has increased. Humpback whales migrate from the Northern hemisphere to the Southern to spend the winter months in warmer waters. Around Panama and Costa Rica, the migration route crosses some of the busiest shipping lanes in the world. Dr Hector Guzman from the Smithsonian Tropical Research Institute in Panama City has been tagging humpback whales with transmitters that can be monitored by
satellite. He has been able to overlay the routes taken by individual whales with those taken by individual ships. These tracks show that typically an individual humpback comes within 200 m of a ship at least once a day. This data will enable shipping routes to be modified to reduce the frequency of collisions.

Watching whales close up is a most exciting and thought\sphinxhyphen{}provoking experience. Seeing the great whales in their natural habitat, you can be awed by their size. A blue whale skeleton hangs in a natural pose in a London museum, inciting wonder in visitors and demonstrating the size of the biggest whale species \textendash{} as well as the biggest animal that has ever existed.

\sphinxincludegraphics[width=880\sphinxpxdimen,height=660\sphinxpxdimen]{{s182_11_figure_18_blue_whale_skeleton_alt}.jpg}

Figure 20 Skeleton of a blue whale, \sphinxstyleemphasis{Balaenoptera musculus}, hanging in the Natural History Museum in London, UK


\subsection{Conclusion}
\label{\detokenize{content/session_00/Part_00_06:Conclusion}}\label{\detokenize{content/session_00/Part_00_06::doc}}
This free course provided an introduction to studying aquatic mammals. You have now learned about a variety of aquatic mammals, their behaviours and their evolutionary history. The final section then introduced the ways in which humans have impacted the ecosystem.

The course activities were designed to help you to improve your confidence as an independent learner, and develop your approach to study and learning at a distance.

If you’d like to learn more about the oceans, visit our \sphinxhref{https://www.open.edu/openlearn/oceanexplorer}{Ocean Explorer} interactive, and take a look at some other free study materials on that page.

\sphinxincludegraphics[width=512\sphinxpxdimen,height=288\sphinxpxdimen]{{oceans-promo}.png}

Ocean Explorer

To find out a bit more about aquatic life, you can download our \sphinxhref{https://www.open.edu/openlearn/nature-environment/natural-history/download-your-free-oceans-poster}{free Oceans poster} that accompanied Blue Planet, the OU/BBC TV series.


\subsection{Keep on learning}
\label{\detokenize{content/session_00/Part_00_07:Keep-on-learning}}\label{\detokenize{content/session_00/Part_00_07::doc}}
\sphinxincludegraphics[width=300\sphinxpxdimen,height=200\sphinxpxdimen]{{ol_skeleton_keeponlearning_image}.jpg}


\bigskip\hrule\bigskip



\subsubsection{Study another free course}
\label{\detokenize{content/session_00/Part_00_07:Study-another-free-course}}
There are more than \sphinxstylestrong{900 courses on OpenLearn} for you to choose from on a range of subjects.

Find out more about all our \sphinxhref{http://www.open.edu/openlearn/free-courses?LKCAMPAIGN=OLSU\_KeepLearning\&MEDIA=\_OL}{free courses}.


\bigskip\hrule\bigskip



\bigskip\hrule\bigskip



\subsubsection{Take your studies further}
\label{\detokenize{content/session_00/Part_00_07:Take-your-studies-further}}
Find out more about studying with The Open University by \sphinxhref{http://www.open.ac.uk/courses?LKCAMPAIGN=OLSU\_KeepLearning\&MEDIA=\_OU}{visiting our online prospectus}.

If you are new to university study, you may be interested in our \sphinxhref{http://www.open.ac.uk/courses/do-it/access?LKCAMPAIGN=OLSU\_KeepLearning\&MEDIA=\_OU}{Access Courses} or \sphinxhref{http://www.open.ac.uk/courses/certificates-he?LKCAMPAIGN=OLSU\_KeepLearning\&MEDIA=\_OU}{Certificates}.


\bigskip\hrule\bigskip



\bigskip\hrule\bigskip



\subsubsection{What’s new from OpenLearn?}
\label{\detokenize{content/session_00/Part_00_07:What_u2019s-new-from-OpenLearn?}}
\sphinxhref{http://www.open.edu/openlearn/about-openlearn/subscribe-the-openlearn-newsletter?LKCAMPAIGN=OLSU\_KeepLearning\&MEDIA=\_OL}{Sign up to our newsletter} or view a sample.


\bigskip\hrule\bigskip


For reference, full URLs to pages listed above:

OpenLearn \sphinxhyphen{} \sphinxhref{http://www.open.edu/openlearn/free-courses?LKCAMPAIGN=OLSU\_KeepLearning\&MEDIA=\_OL}{www.open.edu/openlearn/free\sphinxhyphen{}courses}

Visiting our online prospectus \sphinxhyphen{} \sphinxhref{http://www.open.ac.uk/courses?LKCAMPAIGN=OLSU\_KeepLearning\&MEDIA=\_OU}{www.open.ac.uk/courses}

Access Courses \sphinxhyphen{} \sphinxhref{http://www.open.ac.uk/courses/do-it/access?LKCAMPAIGN=OLSU\_KeepLearning\&MEDIA=\_OU}{www.open.ac.uk/courses/do\sphinxhyphen{}it/access}

Certificates \sphinxhyphen{} \sphinxhref{http://www.open.ac.uk/courses/certificates-he?LKCAMPAIGN=OLSU\_KeepLearning\&MEDIA=\_OU}{www.open.ac.uk/courses/certificates\sphinxhyphen{}he}

Newsletter \sphinxhyphen{} \sphinxhref{http://www.open.edu/openlearn/about-openlearn/subscribe-the-openlearn-newsletter?LKCAMPAIGN=OLSU\_KeepLearning\&MEDIA=\_OL}{www.open.edu/openlearn/about\sphinxhyphen{}openlearn/subscribe\sphinxhyphen{}the\sphinxhyphen{}openlearn\sphinxhyphen{}newsletter}



\renewcommand{\indexname}{Index}
\printindex
\end{document}